\documentclass[11pt]{article}

\usepackage{amsmath}

\usepackage{amsfonts,amssymb,verbatim}

\usepackage{amsthm}

\usepackage{mathrsfs}

\usepackage{graphicx}

\usepackage{colortbl,hhline}

\usepackage{fullpage}

\usepackage{calc}

\usepackage[francais]{babel}

\usepackage{fancyhdr}

\usepackage{dsfont}

\usepackage[latin1]{inputenc}

\usepackage{xcolor}

\usepackage{xcolor,rotating,epic,eepic}

\usepackage{tikz-qtree}

\usetikzlibrary{matrix}

 

\usepackage{fancyhdr}

\usepackage{xcolor,rotating,epic,eepic}

\usepackage{tikz}

\usepackage[babel=true,kerning=true]{microtype}

 

\usetikzlibrary{%

  arrows,%

  calc,%

  shapes.geometric,%

  shapes.misc,%

  shapes.symbols,%

  shapes.arrows,%

  automata,%

  through,%

  positioning,%

  scopes,%

  decorations.shapes,%

  decorations.text,%

  decorations.pathmorphing,%

  shadows}

 

\begin{document}

 

\begin{figure}[h]

   \centering

\begin{tikzpicture}[>=stealth,sloped]

 

    \matrix (tree) [%

      matrix of nodes,

      minimum size=0.25cm,

      column sep=0.25cm,

      row sep=1.5cm,

    ]

    {                                                                                                    

                                                                                           

          &   &   &\fbox{\begin{minipage}[t]{0.1\textwidth}\begin{center}Computer\end{center} \end{minipage}}&   &   &  \\

          & \fbox{\begin{minipage}[t]{0.1\textwidth}\begin{center}Desktop\end{center} \end{minipage}} &   &  

 &   & \fbox{\begin{minipage}[t]{0.1\textwidth}\begin{center}Laptop\end{center} \end{minipage}}  &  \\

        \fbox{\begin{minipage}[t]{0.1\textwidth}\begin{center}HP\end{center} \end{minipage}} & 

  & \fbox{\begin{minipage}[t]{0.1\textwidth}\begin{center}IBM\end{center} \end{minipage}} &  

 & \fbox{\begin{minipage}[t]{0.1\textwidth}\begin{center}DELL\end{center} \end{minipage}} & 

   &\fbox{\begin{minipage}[t]{0.1\textwidth}\begin{center}HP\end{center} \end{minipage}}\\

    };

   

    \draw[->] (tree-1-4) -- (tree-2-2) node [midway,above]{}; %{$P$};

    \draw[->] (tree-1-4) -- (tree-2-6) node [midway,below]{}; %{$(1-p)$};

    

    \draw[->] (tree-2-2) -- (tree-3-1) node [midway,above]{}; %{$P^2$};

    \draw[->] (tree-2-2) -- (tree-3-3) node [midway,below]{}; %{$(1-p)p$};

    

    \draw[->] (tree-2-6) -- (tree-3-7) node [midway,above]{}; %{$(1-p)p$};

    \draw[->] (tree-2-6) -- (tree-3-5) node [midway,below]{}; %{$(1-p)^2$}; 

    

\end{tikzpicture}

\caption{\textbf{\label {}The multilevel of the abstraction}} 

\end{figure}

\end{document}

